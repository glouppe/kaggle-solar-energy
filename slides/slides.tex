\documentclass[handout]{beamer}

\usepackage[utf8]{inputenc}
\usepackage[frenchb]{babel}
\usepackage{verbatim}
\usepackage{graphicx}
\usepackage{color}
\usepackage{hyperref}
\usepackage{verbatim}
\usepackage{url}

\hypersetup{colorlinks=true, linkcolor=black, urlcolor=blue}
\usetheme{boxes}
\beamertemplatenavigationsymbolsempty
\setbeamertemplate{sections/subsections in toc}[circle]

\usepackage{pifont}
\newcommand{\cmark}{\ding{51}}%
\newcommand{\xmark}{\ding{55}}%

\title{Forecasting Daily Solar Energy Production Using Robust Regression Techniques}
% \author{Gilles Louppe and Peter Prettenhofer}
% \institute{Université de Liège, Belgium\\
% Graz University of Technology, Austria}
% \date{February, 2014}
\date{}

\begin{document}


% Title slide =================================================================

\begin{frame}
\titlepage

\vspace{-2cm}

\begin{columns}[T]
\begin{column}{.48\textwidth}

\centering
Gilles Louppe (\href{https://twitter.com/glouppe}{@glouppe})\\
{\small\it Université de Liège, Belgium}

\end{column}
\begin{column}{.48\textwidth}

\centering
Peter Prettenhofer (\href{https://twitter.com/pprett}{@pprett})\\
{\small\it Graz University of Technology, Austria}

\end{column}
\end{columns}


\end{frame}


% Slide 1 =====================================================================

\begin{frame}{Problem statement}
  \begin{block}{Goal}
      Short-term forecasting of daily solar energy production based on weather forecasts from numerical weather prediction (NWP) models.
  \end{block}


\begin{columns}[T]
\begin{column}{.45\textwidth}

  \begin{block}{Challenges}
      \begin{itemize}
         \item High volatility \\{\tiny rapidly changing weather conditions}

         \item Noisy response \\ {\tiny hardware failure}
         \item Noisy inputs \\ {\tiny inaccuracy of NWP model}
      \end{itemize}
  \end{block}

\end{column}
\begin{column}{.45\textwidth}
  \begin{figure}
    \includegraphics[width=\textwidth]{images/volatility.pdf}
  \end{figure}

\end{column}
\end{columns}
\end{frame}


% Slide 2 =====================================================================

\begin{frame}{Data}

\begin{columns}[T]
    \begin{column}{.45\textwidth}

\begin{block}{Solar energy production}
  \begin{itemize}
     \item 98 Oklahoma Mesonet sites
     \item Total incoming solar energy in $J m^{-2}$
     \item Time period: 1994 - 2007
  \end{itemize}
\end{block}

    \end{column}
    \begin{column}{.45\textwidth}

  \begin{figure}
    \includegraphics[width=\textwidth]{images/gefs_mesonet_stations.png}\\
    {\color{gray}\tiny{Courtesy: Dr. Amy McGovern}}
  \end{figure}

    \end{column}
  \end{columns}

\begin{block}{Numerical weather prediction}
\begin{itemize}
     \item NOAA/NCEP GEFS Reforecast, 5 forecasts per day
     \item Ensemble comprises 11 members (one control)
     \item 15 measurements (temp, humidity, upward radiative flux, ...)
  \end{itemize}
\end{block}

\end{frame}


% Slide 3 =====================================================================

\begin{frame}{Overview of our approach}

\begin{figure}
    \centering
    \includegraphics[width=0.9\textwidth]{images/process.pdf}
\end{figure}

\vspace{-2cm}

\begin{enumerate}
\item \textbf{Interpolation} of meteorological measurements from GEFS grid points onto Mesonet sites;
\item Construction of \textbf{new variables} from the measurement estimates;
\item \textbf{Forecasting} of daily energy production using Gradient Boosted Regression Trees.
\end{enumerate}

\end{frame}


% Slide 4 =====================================================================

\begin{frame}{Kriging}

\textbf{Goal:} Obtain local meteorological measurements at Mesonet sites.

\vskip0.25cm

For each day $d$, period $h$ and type $f$ of meteorological measurement:

\begin{enumerate}

\item Build a local learning set $${\cal L}_{dhf} = \{ (\mathbf{x}_i =
(\text{lat}_i, \text{lon}_i, \text{elevation}_i), y_i = \overline{m_{idhf}} )
\},$$  where $\overline{m_{idhf}}$ is the average value (over the ensemble) of
measurements $m_{idhf}$ of type $f$, at GEFS location $i$, day $d$ and period
$h$;

\item Learn a Gaussian Process from ${\cal L}_{dhf}$, for predicting
measurements from coordinates;\\
{\scriptsize(Fitting is perfomed using \textit{nuggets} to account for noise in the measurements.)}

\item Predict measurement estimates $\widehat{m_{jdhf}}$ at Mesonet stations
$j$ from their coordinates.

\end{enumerate}

\end{frame}

%% NOTE: to illustrate Kriging interpolation we can use this plot: https://www.dropbox.com/s/as791zl9w166ihn/kriging_nugget_interpolation.png


% Slide 5 =====================================================================

\begin{frame}{Feature engineering}

\textbf{Goal:} Build a learning set ${\cal L}$ from the measurement estimates.

\begin{enumerate}

\item Concatenate the estimates at all periods $h$ and for all types $f$, for
each Mesonet station $j$ and day $d$:
$${\cal L} = \{ (\mathbf{x}_{jd} = (\widehat{m_{jd{h_1}{f_1}}}, \widehat{m_{jd{h_1}{f_2}}}, ...), y_{jd} = p_{jd}) \}$$
where $p_{jd}$ is the energy production at Mesonet station $j$ and day $d$.

\item Extend inputs $\mathbf{x}_{jd}$ with engineered features:
\begin{itemize}
\item Solar features (delta between sunrise and sunset)
\item Temporal features (day of year, month)
\item Spatial features (latitude, longitude, elevation)
\item Non-linear combinations of measurement estimates
\item Daily mean estimates
\item Variance of the measurement estimates, as produced by the Gaussian Processes
\end{itemize}

\end{enumerate}

\end{frame}


% Slide 6 =====================================================================

\begin{frame}{Predicting energy production}

\textbf{Goal:} Predict daily energy production at Mesonet sites.

\begin{enumerate}
\item Learn a model using Gradient Boosted Regression Trees (\texttt{sklearn.ensemble.GradientBoostingRegressor}), predicting output $y$ from inputs $\mathbf{x}$;
    \begin{itemize}
        \item Optimize hyper-parameters on an internal validation set;
        \item Use the \textit{Least Absolute Deviation} loss for robustness;
    \end{itemize}
\item For further robustness, repeat Step 1 several times (using different random seeds) and aggregate the predictions of all models.
\end{enumerate}

\end{frame}


% Slide 7 =====================================================================

\begin{frame}{Results}
  \begin{block}{Evaluation}
      \begin{itemize}
          \item Held-out data from 2008 - 2012.
          \item Mean Absolute Error (MAE) as metric:\\
$${MAE} = \frac{1}{JD} \sum_{j=1}^{J} \sum_{d=1}^{D} | p_{jd} - \hat{p}_{jd} | $$
      \end{itemize}
  \end{block}
  \begin{block}{Results}
        \vspace*{0.4cm}
        \begin{tabular}{ l c r }
          \textbf{Method} & \textbf{Heldout-Score [MAE]} & \textbf{$\Delta$ [\%]} \\
          GMM & 4019469.94 & 46.19\% \\
          Spline Interp. & 2611293.30 & 17.17\% \\
          Kriging + GBRT & 2162799.74 & - \\
          Best & 2107588.17 & -2.62\% \\
        \end{tabular}
  \end{block}
\end{frame}

% Slide 8 =====================================================================

\begin{frame}{Error analysis}

\begin{figure}
    \centering
    \hspace*{-0.9cm}
    \includegraphics[width=1.15\textwidth]{images/hk_11_err1.pdf}
\end{figure}


\end{frame}

% Slide 9 =====================================================================

\begin{frame}{Conclusions}

{\color{green} \cmark} \textbf{Competitive} results (4th position);

\vspace{0.5cm}

{\color{green} \cmark} \textbf{Cautious} and \textbf{robust} approach at all steps of the pipeline;

\vspace{0.5cm}

{\color{red} \xmark} Including additional data from nearest GEFS grid points might have further improved our results.

\end{frame}


% Backup Slides =====================================================================
\begin{frame}

\centering
\it Backup Slides

\end{frame}


% Illustration of Kriging ===========================================================
\begin{frame}{Kriging illustration}

\begin{figure}
    \includegraphics[scale=0.35]{images/kriging_nugget_interpolation.png}
\end{figure}

\end{frame}


\end{document}
